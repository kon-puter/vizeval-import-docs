V dnešní době se stala systematická a průběžná autoevaluace povinností škol.
Proces autoevaluace muže být zdlouhavý a nepříjemný. Proces autoevaluace by měl pomoct zlepšit kvalitu výuky.\cite{sbornik-autoevaluace} 
Tento nástroj přispívá k řešení tohoto problému formou dotazníku.
Studenti hodnotí úvazky, neboli kombinaci učitel-předmět. 
Učitelé hodnotí předmět a~skupinu/skupiny, které učí. 
Nástroj podporuje číselné otázky a textové odpovědi, které jdou nastavit jako skryté pro učitele.
Odpovědi je možné archivovat mezi více školními roky a během jednoho školního roku je možnost více hodnotících období a více importů.
Tato část dokumentace se věnuje importu.

\section{Motivace}
Úplná komplexní autoevaluace je časově náročná a neoblíbená aktivita.\cite{sbornik-autoevaluace}
Cílem tohoto nástroje je snížit administrativní zátěž při autoevaluaci.
Elektronické shromažďování odpovědí skrz webový portál eliminuje nutnost digitalizace papírových odpovědí tradičního hlasování. 
Nástroj automaticky provede základní vizualizace výsledků a při nutnosti složitější analýzy poskytne strukturovaný výstup vhodný pro tabulkové procesory.

Další výhodou tohoto způsobu autoevaluace je vyšší anonymita\footnote{Nikde se neukládá obsah odpovědi uživatelů} než u papírových anket, nejde najít autor podle rukopisu.  
Výsledky a vizualizace se aktualizují v realném čase podle odpovědi respondentů a umožňují předběžný náhled na situaci.

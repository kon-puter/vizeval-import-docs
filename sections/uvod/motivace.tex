V dnešní době se stala systematická a průběžná autoevaluace povinností škol.
Proces autoevaluace muže být zdlouhavý a nepříjemný.\cite{sbornik-autoevaluace} 
Tento nástroj přispívá k řešení tohoto problému formou dotazníku.
Studenti hodnotí úvazky, neboli kombinaci učitel-předmět. 
Učitelé hodnotí předmět a skupinu/skupiny, které učí. 
Nástroj podporuje číselné otázky a 

\section{Motivace}
\unsure{Odsazení odstavců ?}
Úplná komplexní autoevaluace je časově náročná a často neoblíbená aktivita.\cite{sbornik-autoevaluace}
Cílem tohoto nástroje je snížit administrativní zátěž při autoevaluaci.
Elektronické shromažďování odpovědí skrz webový portál eliminuje nutnost digitalizace papírových odpovědí. 
Nástroj automaticky provede základní vizualizace výsledků a při nutnosti složitější analýzy poskytne strukturovaný výstup vhodný pro tabulkové procesory.

Další výhodou tohoto způsobu autoevaluace je vyšší anonymita\footnote{Nikde se neukládá obsah odpovědi uživatelů} než u papírových anket, nejde najít autor podle rukopisu.  
Výsledky a vizualizace se aktualizují v realném čase podle odpovědi respondentů a umožňují předběžný náhled na situaci.



Bez tohoto nástroje bylo potřeba hlasovat o kvalitě výuky na papír.
O něco lepší možností by bylo použít nějaké dostupné online formuláře.
Tam je však obtížné zajistit, aby mohli hlasovat o úvazku jen vybraní studenti a učitelé.
Další problém je že formulářů by bylo opravdu velké množství (pro naší školu zhruba 450).
Proto vzniká tento evaluační nástroj.
Využije existující data ze školních systému.
Poskytne zpětnou vazbu učitelům a umožní jim tak zlepšit kvalitu výuky.

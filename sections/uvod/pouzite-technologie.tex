\section{Použité technologie}
\improvement{pořád se opakuje používáme, první osoba na popis postupu }

Používáme framework \href{https://symfony.com/}{Symfony}.
Symfony je open-source PHP framework, který je vydán pod MIT licencí.
Využívá návrhový vzor MVC a moderní PHP.
Symfony je modulární a jednoduše rozšiřitelný pomocí existujících bundlů/componentů\footnote, klade důraz na možnost konfigurace a standardizaci.
To je jeden z hlavních důvodů, proč byl tento framework vybrán.
Používá ho přes 600000 vývojářů a má 713M stažení měsíčně.\cite{symfony:what-is}
Již spoustu problémů řeší za nás a zamezuje tím bezpečnostním a jiným chybám.


\subsection{Databáze}
Jako persistance vrstvu používáme Doctrine ORM, to nám poskytuje řadu výhod jako typesafe mapování, nezávislost na databázi\footnote{Doctrine ORM aktuálně podporuje PostgreSQL, MySQL, Oracle, SQL server a další\cite{doctrine-supported-dbs}}.
S Doctrine používáme přístup code first, napíšeme standardní PHP třídu, kde pomocí atributů přidáme vlastnostem informace o jejich mapování. 

Pro uchovávání dat používáme relační databázi PostgreSQL, ta podporuje pokročilejší funkce jako transakce DDL narozdíl od MySQL.  


\section{Symfony EasyAdmin}

Jedná se o bundle, který je určen k vytváření administrací.
Vstupním bodem takové administrace je Dashboard, který obsahuje další stránky, těmi můžou být CRUD stránky EasyAdminu nebo vlastní stránka.

Síla EasyAdminu spočívá v integraci společně s Doctrine.
Dokáže vytvořit pro danou entitu CRUD stránku, která je jednoduše upravitelná a přizpůsobitelná.
EasyAdmin poskytuje důležité prvky jako stránkování a filtry.
Poskytuje možnost doplnění o nové akce. Akce nějakým způsobem upravují entitu/entity.

EasyAdmin umožňuje změnit vzhled přepsáním výchozích templatů.


\subsection{Serverové technologie}
Jako web server používáme FrankenPHP, moderní php server založený na serveru Caddy.
FrankenPHP podporuje worker mode, který umožňuje znovu použití PHP interpretoru při dalším requestu.

Naše aplikace je nasazována pomocí dockeru. Pro každou školu máme zvlášť container s databázovým a webovým serverem.
Jako reverse proxy používáme Traefic, který nám umožňuje jednoduchou správu instancí.

\subsection{Frontendové technologie}

U importu používám balíky z iniciativy \href{https://ux.symfony.com/}{Symfony UX} jako forntendové řešení.
Celý import obsahuje úplné minimum JavaScriptu, ten je navíc ve Stimulus controllerech. Označí se HTML element, na který chci aby se kontroler připojil a potom mám přístup k targetům daného kontroleru.
Taky používám Hotwire Turbo\label{ux-turbo}, které jednoduše napodobuje funkcionalitu SPA, Turbo funguje skvěle se Stimulusem.
Na některé části používám LiveComponent, což je něco jako Phoenix LiveWire, kde je uživatelské rozhraní řízeno přímo ze strany serveru. 
Některé formuláře jsou sestavované pomocí Symfony Form, ten se potom stará o mapování vstupu a validaci\footnote{Na validaci používá Symfony Validator.}.
Symfony Form vygeneruje přímo HTML form se všemi potřebnými náležitostmi jako zpracování chyb a stylování\footnote{Jako styl je možno specifikovat template. My používáme Bootstrap}, taky zamezuje XSRF útokům\footnote{Do formuláře vkládá XSRF token, který potom validuje.}.

  \subsection{Konečný automat}\label{uvod:konecny-automat}

Je abstraktní automat, který může být v přávě jednom stavu.\cite{wiki:finite-state-machine-en} Když automat přechází z jednoho stavu do druhého nazývá se to přechod.
Automat se nazývá konečným, protože má jen konečný počet stavů.
Využívají se pro parsování formálních jazyků a jsou užitečné při modelování logiky aplikace.
Mezi jiná využití patří parsování regularních výrazů.\cite{wiki:finite-state-machine-cz}

Jako implementaci konečného automatu používám Symfony Workflow.
Taky je konečný automat používán při zvýrazňování výrazů pro generování emailů v editoru Ace.



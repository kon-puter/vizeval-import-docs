\section{Použité technologie}

Používáme framework \href{https://symfony.com/}{Symfony}. Symfony je open-source PHP framework, který je vydán pod MIT licencí.  Využívá návrhový vzor MVC a moderní PHP. Symfony je modulární a jednoduše rozšiřitelný pomocí existujících bundlů/componentů. To je jeden z hlavních důvodů, proč byl tento framework vybrán. Již spoustu problémů řeší za nás a zamezuje tím bezpečnostním a jiným chybám.

\subsection{Databáze}
Používáme Symfony společně s Doctrine ORM, to nám poskytuje řadu výhod jako typesafe mapování, nezávislost na databázi\footnote{Doctrine ORM aktuálně podporuje PostgreSQL, MySQL, Oracle, SQL server a další\cite{doctrine-supported-dbs}}. S Doctrine používáme přístup code first, napíšeme standardní PHP třídu, kde pomocí atributů přidáme vlastnostem informace o jejich mapování. 
\unsure{Pridat example}
Pro uchovávání dat používáme relační databázi PostgreSQL, ta podporuje  pokročilejší funkce jako transakce DDL narozdíl od MySQL.  

\subsection{Symfony EasyAdmin}
Za zmíňku stojí Symfony EasyAdmin, který dokáže vytvořit pro danou entitu CRUD stránku, která je jednoduše upravitelná.

\subsection{Serverové technologie}
Jako web server používáme FrankenPHP, moderní php server založený na serveru Caddy. FrankenPHP podporuje worker mode, který umožňuje znovu použití PHP interpretoru při dalším requestu.

Naše aplikace je nasazována pomocí dockeru. Pro každou školu máme zvlášť container s databázovým a webovým serverem. Jako reverse proxy používáme Traefic, který nám umožňuje jednoduchou správu instancí.

\section{Motivace}
\unsure{Odsazení odstavců ?}
Tento nástroj se snaží zjistit úroveň vzdělávání pomocí anket směřovaných na učitele a žáky. Bez tohoto nástroje bylo potřeba hlasovat o kvalitě výuky na papír. O něco lepší možností by bylo použít nějaké dostupné online formuláře. Tam je však obtížné zajistit, aby mohli hlasovat o úvazku jen vybraní studenti a učitelé. Další problém je že formulářů by bylo opravdu velké množství (pro naší školu zhruba 450). Proto vzniká tento evaluační nástroj. Využije existující data ze školních systému. Poskytne zpětnou vazbu učitelům a umožní jim tak zlepšit kvalitu výuky. Za tímto účelem jsou v administračním a učitelském webovém portálu poskytnuty vizualizace výsledků.

\section{Cíle}
Shrnutí cílů a požadavků.
\begin{itemize}
    \item Jednoduché vytvoření testování
    \item Anonymní hlasování
    \item Přehledná vizualizace hodnocení
    \item Snadný import ze školního informačního systému
    \item Možnost importu pro co nejvíce škol
    \item Zabezpečení dat
\end{itemize}

\section{Podobné aplikace}
\subsection{Předchozí verze}\label{uvod:predchozi-verze}
Za zmíňku stojí řešení jednoho z bývalých studentů Jakuba Šenka. Ten také vytvořil evaluační nástroj, který byl jednu dobu používán na naší škole. Měl však řadu neduh. Unikaly výsledky hodnocení a studenti mohli hodnotit jeden úvazek více krát. \improvement{další info} Navíc nebyl jednoduše rozšířitelný, protože nepoužíval žádný framework. Náš projekt neobsahuje žádný společný kód.
\subsection{Studentská anketa Univerzity Karlovy}
Karlova univerzita vyvinula systém studentských anket, který je úzce provázaný s jejich vlastním informačním systémem SIS.  Jeden z rozdílů oproti našemu řešení je možnost podepsat odpovědi.\cite{matfyz-anketa}



